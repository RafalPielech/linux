\documentclass[11pt,a4paper]{article}
\usepackage{polski}
\usepackage[utf8]{inputenc}
\title{Memory Technology Device (MTD)}
\author{Rafał Pielech}
\date{}
 
\begin{document}
\maketitle
\section{Notatki}
Tutaj kilka obserwacji dot. MTD.
\subsection{Dodawanie partycji}
\verb1mtd_device_parse_register(...)1 - rejestruje partycje poprzez wywołanie \verb1parse_mtd_partitions1. Dostaje ona listę partycji (typu \verb1mtd_partition1) wcześniej przygotowaną przez sterowniki poszczególnych kości pamięci. Struktura \verb1mtd_inf1 jak i \verb1mtd_partition1 dają pełną wiedzę dla MTD do używania pamięci.

Funkcja \verb1mtd_device_parse_register(...)1 nie zawsze dostaje gotową listę partycji. Czasami sterownik pamięci przesyła \verb1of_node1 jako parser data (trzeci parametr). Dzieje się tak w przypadku \verb8mpc51218. Wtedy sterownik podaje \verb1mtd_info1 ze wskaźnikami do funkcji obsługujących pamięć i wspomniany \verb1of_node1.

Partycje potem są alokowane w funkcji \verb1allocate_partition1. Funkcja ta bierze master, czyli rodzica partycji, urządzenie zawierające alokowaną partycję. Potem partycję, jej numer i offset. Zwracany jest zaalokowany obiekt typu \verb1mtd_part1, który dużo dziedziczy po masterze, natomiast funkcje czytające, piszące dla owej partycji nie są bezpośrednio dziedziczone. Tam przypisywane są funkcje z prefiksem \verb1part1, które i tak odwołują się do funkcji mastera w swoim ciele.
\end{document}
