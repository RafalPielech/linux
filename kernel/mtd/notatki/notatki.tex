\documentclass[11pt]{article}
\usepackage{geometry}
\geometry{a4paper, total={165mm,257mm}, left=20mm, top=20mm}
\usepackage{polski}
\usepackage[utf8]{inputenc}
\title{Memory Technology Device (MTD)}
\author{Rafał Pielech}
 
\begin{document}
\maketitle
\section{Notatki}
Tutaj kilka obserwacji dot. MTD.
\subsection{Dodawanie partycji}
\verb1mtd_device_parse_register1 rejestruje partycje poprzez wywołanie funkcji \verb1parse_mtd_partitions1. Dostaje ona listę partycji (typu \verb1mtd_partition1) wcześniej przygotowaną przez sterownik kości pamięci. Struktura \verb1mtd_info1 jak i \verb1mtd_partition1 dają pełną wiedzę dla MTD do używania pamięci. Funkcja \verb1mtd_device_parse_register1 nie zawsze dostaje gotową listę partycji. Czasami sterownik pamięci przesyła \verb1of_node1 jako parser data (trzeci parametr). Dzieje się tak w przypadku mpc5121. Wtedy sterownik podaje \verb1mtd_info1 ze wskaźnikami do funkcji obsługujących pamięć i wspomniany \verb1of_node1.

Partycje potem są alokowane w funkcji \verb1allocate_partition1. Funkcja ta bierze master, czyli rodzi-
ca partycji - urządzenie zawierające alokowaną partycję. Potem partycję, jej numer i offset. Zwracany
jest zaalokowany obiekt typu \verb1mtd_part1, który dużo dziedziczy po masterze, natomiast funkcje czy-
tające, piszące dla owej partycji nie są bezpośrednio dziedziczone. Tam przypisywane są funkcje z
prefiksem part, które i tak odwołują się do funkcji mastera w swoim ciele.

\subsection{Podsystem UBI}
W pliku \verb1drivers/mtd/ubi/build.c1 w funcji \verb1ubi_init1 inicjalizowane są wolumeny UBI zgodnie z podanymi parametrami modułu. Funkcja \verb1ubi_attach_mtd_dev1 mając podane \verb1mtd_info1, offset nagłówków \verb1ubi_ec_hdr1 i \verb1ubi_vid_hdr1 (\cite{ubidesign} strona 15) (te dwa dodawane są zawsze do każdego poprawnego erase block'u) i maksymalną liczbę PEB-ów (Physical Erase Block) na 1024 sztuki, tworzy blok UBI w statycznej tablicy modułu. Wcześniej tworzy odpowiednią klasę dla sysfs. Na spodzie pliku \verb1drivers/mtd/ubi/build.c1 jest pełen opis parametrów modułu, które są parsowane.

Nagłówek \verb1ubi_ec_hdr1 jest aktualizowany po każdym wykasowaniu bloku. Jeżeli nastąpi utrata zasilania po wykasowaniu bloku to nagłówek nie zostanie zaktualizowany i potem wpsana mu zostania wartość uśredniona. Tłumaczy to \cite{ubidesign} strona 16.

UBI dokonuje logicznego przemapowania najbardziej używanych bloków fizycznych. Jeżeli dany blok logiczny jest często używany to UBI podpisuje mu co pewien interwał inny blok fizyczny dbając o równomierne zużywanie flasha \cite{ubidesign} strona 12, \textit{Also, if a logical eraseblock is used too actively and UBI decides
to assign it to another (less worn out) physical eraseblock\ldots} . \marginpar{\tiny{Wear leveling}}

\subsection{Tworzenie wolumenów}
Są narzędzia UBI utils, które pomagają tworzyć wolumeny. W jądrze jest obsługa instrukcji ioctl \verb1UBI_IOCMKVOL1, która tworzy volumen dla danego UBI zgodnie z podanymi parametrami \verb1ubi_mkvol_req1 podanymi przez użytkownika posługując się funkcją \verb1ubi_create_volume1. Dodaje strukturę \verb1ubi_volume1 do tablicy wewnątrz wybranego \verb1ubi_device1. Funkcja \verb1ubi_create_volume1 tworzy listę atrybutów dla tworzonego wolumenu za pomocą funkcji \verb1volume_sysfs_init1.

\subsection{UBIFS}
UBI file system bezpośrednio odwołuje się do wolumenów UBI podczas gdy jest montowany w funkcji \verb1ubifs_mount1 w pliku \verb1kernel/fs/ubifs/super.c1. Ta funkcja montująca zwraca wskaźnik do struktury \verb1dentry1, która jest przetrzymywana przez pole \verb1s_root1 struktury \verb1super_block1 zaalokowanej i wypełnionej (funkcja \verb1ubifs_fill_super1) tej funcji.



\verb1ubiconcat1 może połączyć partycje wielu urządzeń fizycznych w jeden volumen UBI.

\begin{thebibliography}{9}
\bibitem{ubidesign} 
UBI - Unsorted Block Images,\\
\texttt{http://www.dubeiko.com/development/FileSystems/UBI/ubidesign.pdf}
\end{thebibliography}

\end{document}